\documentclass{article}

\begin{document}

\title{Math 130, 1.10 Problem #13}

\author{Martin Camacho}
\maketitle

Solve the following initial value problem:
\[ 
  y' = \frac{-2xy}{1 + x^2} \quad
  y(0) = 2  \quad h = 0.05 \quad \textrm{and} \quad y(0.5)
\]

We know this is a first-order linear ODE. But how can this be solved? 
Let's try to make it seprable:

\[ \frac{dy}{dx} = \frac{-2xy}{1 + x^2} \]
\[ \frac{dy}{dx}(1 + x^2) = -2xy \]
\[ \frac{1}{y}dy = \frac{-2x}{1 + x^2}dx \]

Now that we have all the \(y\)'s on the left and the \(x\)'s on the right, we can integrate:

\[ \int \frac{1}{y} \,dy = \int \frac{-2x}{1 + x^2} \,dx \]

The left side is trivial:

\[ \int \frac{1}{y} = \ln(y) + C_1\]

The right side can be solved with substitution:

\[\textrm{Let} \quad u = x^2 + 1 \quad du = 2xdx \quad dx = \frac{du}{2x}\]

Now do the substitution:

\[-\int \frac{2x}{1 + x^2} \,dx = -\int \frac{2x}{u} \,\frac{du}{2x} = -\int \frac{1}{u} \,du = -\ln(u) + C_2 = -\ln(x^2 + 1) + C_2\]

Putting the result of both integrations together:

\[ \ln(y) + C_1 = -\ln(x^2 + 1) + C_2\]
\[ \ln(y) = -\ln(x^2 + 1) + C_3\]
\[ \ln(y) = \ln((x^2 + 1)^{-1}) + C_3\]
\[ y = e^{\ln((x^2 + 1)^{-1}) + C_3} = e^{\ln((x^2 + 1)^{-1})}e^{C_3}\]

So the general solution is:
\[ y = \frac{e^{C_3}}{X^2 + 1}\]

Using the initial value  \(y(0) = 1\), we can calculate  \(C_3\).

\[ y(0) = 1 = \frac{e^{C_3}}{0^2 + 1} = e^{C_3}\]
\[ C_3 = \ln(1) = 0\]

Therefore, the particular solution is
\[ y = \frac{1}{X^2 + 1}\]



\end{document}